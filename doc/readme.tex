\documentclass[preprint]{aastex}

\usepackage{apjfonts}

\def\mica{{\texttt{MICA}}}

\begin{document}

\title{\texttt{MICA}: A Non-parametric Approach to Constrain the Transfer Function\\ in Reverberation Mapping}
\author{Yan-Rong Li}
\maketitle

\section{Third-party Software Dependencies}

\begin{itemize}
 \item LAPACKE: c version of LAPACK
 \item GSL:  GNU Scientific Library
\end{itemize}

\section{Compiling}
Modify the ``\texttt{Makefile}'' according to your system's configurations, and then type ``\texttt{make}''
to compile it. This will create an executable file called ``\texttt{mica}''.

\mica~supports three types of transfer functions:
\begin{itemize}
 \item a family of Gaussians (default option);
 \item a family of top-hats;
 \item a single top-hat (same as in \texttt{JAVELIN}).
\end{itemize}
Switch on the option ``\texttt{OPTIMIZE += -DJAVELIN}'' and ``\texttt{OPTIMIZE += -DTOPHAT}'' 
in the ``\texttt{Makefile}'' to use the latter two types of transfer functions, respectively.


\section{Running \texttt{MICA}}
In a Linux terminal, type\\
=============================\\
\texttt{./mica ./param.txt}\\
=============================\\
to run \texttt{MICA}. Here, the argument ``param.txt'' specifies some configurations for \texttt{MICA}.
Modify the values in ``param.txt'' for your purposes. Generally, larger ``nmcmc'' gives better Markov chains
and therefore more reliable parameter estimation, but at the cost of more computation time.

\mica~searches over the range of ``nc'' in ``param.txt'' and finds out the best smoothing parameter
``nc'' ($K$ in the paper). ``taulim'' specifies the range of the time lag under consideration. You 
need to input an appropriate range according to your light curves.

\section{Outcomes}
The sub-directory ``\texttt{data/}'' contains the outcomes of \mica.
\begin{itemize}
 \item \texttt{transfer.txt}: the best recovered transfer function.
 \item \texttt{results.txt}: some important results.
 \item \texttt{mcmc*.txt}: the generated Markov chains. \texttt{mcmc\_con.txt} is for recovering the continuum 
 only. \texttt{mcmc\_xx.txt} is for recovering the transfer function, where ``xx''
 represents the number of Gaussians/top-hats.
 \item \texttt{scon.txt}: reconstruction of the continuum generated when only recovering the continuum.
 \item \texttt{sall\_con/line.txt}: reconstruction of both the continuum and line light curves
 generated when recovering the transfer function. Note that \texttt{scon.txt} and \texttt{sall\_con.txt}
 are slightly different, because the latter is further constraint by line light curve.
\end{itemize}

\section{Postprocessing}
\mica~assigns the best estimates of the parameters the mean value of the corresponding Markov chain. 
You can do the statistic estimation by yourself on the Markov chain and store the best values and their 
lower errorbar and upper errorbar in the file ``\texttt{data/par.txt}'' (one line per parameter).
Then set ``flag\_mcmc'' in ``param.txt'' to be ``0'', \mica~will read the file ``\texttt{data/par.txt}''
and calculate the transfer function. In sub-directory \texttt{analysis/}, the python script ``analysis.py''
shows an example to do this.

In sub-directory \texttt{analysis/}, the python script ``lcplot.py'' plots the light curves and the best 
recovered transfer function into a pdf file ``fig.pdf''.

\end{document}

